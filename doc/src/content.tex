\section{Maillages et structures de donn\'ees}

Le template \texttt{finite\_element\_space} est responsable de la
construction du mapping entre les indices des degr\'es de libert\'e
locaux \`a chaque \'element, et les indices des degres de libert\'e
globaux.

Soit $M = (V, E)$ un mesh de triangles conformes. On note:
\begin{itemize}
\item $V = \{v_1, v_2, \dots, v_{N_v}\}$ l'ensemble des vertices du
  mesh $M$.
  
\item $\tau = \{K_1, K_2, \dots, K_{N_K}\}$, o\`u $K_i = \{K_{i,1},
  K_{i,2}, K_{i,3}\}$, $K_{i,j}\in V$ l'ensemble des \'elements du
  mesh $M$. On suppose que $\forall i$ les \'elements de $e_i$ sont tri\'es par
  ordre croissants, c'est-\`a-dire que $K_{i,1} < K_{i,2} < K_{i,3}$.
  
\item $A = \left\{ \{v, w\}\ |\ v,w\in V \text{ et } \exists e\in E
  \text{ tel que } v,w \in e\right\}$. On note $a_i = \{a_{i,1},
  a_{i,2}\}$, $i = 1, \dots, N_{a}$ les \'elements de l'ensemble $A$,
  avec $N_a = \#A$. De m\^eme, on suppose que $\forall i$ les \'elements de
  $a_i$ sont tri\'es par ordre croissants, c'est-\`a-dire
  que $a_{i,1} < a_{i,2}$.
\end{itemize}

D'un point de vue de l'impl\'ementation informatique, l'ensemble $V$
est repr\'esent\'e simplement par l'entier $N_v$. Dans un cadre plus
g\'en\'eral, $V$ peut \^etre repr\'esent\'e par une liste d'entiers
unique ${n_1, \dots, n_{N_v}}$, par exemple si la num\'erotation des
vertices ne commence pas \`a 1, ou que des intervalles manquent.

L'ensemble $E$ est repr\'esent\'e par un tableau \`a double entr\'ees
de dimensions $N_e \times 3$:
\begin{addmargin}[0.5in]{0em}
  \texttt{unsigned int elements[n\_e][3];}
\end{addmargin}
o\`u \texttt{elements[i][j]}$ = e_{i,j}$.

L'ensemble $A$ est repr\'esent\'e par un tableau \`a double entr\'ees
de dimensions $N_a \times 2$:
\begin{addmargin}[0.5in]{0em}
  \texttt{unsigned int edges[n\_a][2];}
\end{addmargin}
o\`u \texttt{edges[i][j]}$ = a\_{i,j}$.


\section{Element fini et degr\'es de libert\'e}
Notons $\hat K$ le triangle de sommets $v_1 = (0,0)$, $v_2 = (1,0)$ et
$v_3 = (0, 1)$. On note $\hat N_v = 3$ le nombre de sommets du
triangle, $\hat N_K = 1$ le nombre de triangle et $\hat N_a = 3$ le
nombre d'arr\^ete du triangle. Cette notation, qui parait superflue,
devient n\'ecessaire lorsque l'on g\'eneralise le formalisme \`a des
mesh de simplexes de dimensions $D \neq 2$.

Dans ce cas, l'ensemble $A$ est donn\'e par 
\begin{equation}
  \tau = \left\{ \{v_1, v_2\}, \{v_1, v_3\}, \{v_2, v_3\} \right\}.
\end{equation}

Un \'element fini d\'efinit un certain nombre de degr\'es de
libert\'es sur chaque \'element. En particulier, on note $\hat M_v$ le
nombre de degr\'e de libert\'e associ\'es \`a chaque vertex de $K$,
$\hat M_a$ le nombre de dof associ\'es \`a chaque arr\^ete de $K$, et
$\hat M_K$ le nombre de dof associ\'es \`a l'\'element $K$. Donc le
nombre de dof $\hat M_\text{tot}$ associ\'es \`a $K$ au total est donn\'e par 
\begin{equation}
  \hat M_\text{tot} = \hat M_v \hat N_v + \hat M_a \hat N_a + \hat M_K \hat N_K.
\end{equation}
Dans un mesh les degr\'es de libert\'es associ\'es aux vertices et aux
arr\^etes d'un \'element sont identifi\'es aux degr\'es de libert\'e
associ\'es a un autre \'element qui partage le m\^eme vertex ou la
m\^eme arr\^ete. Ces dofs sont donc associ\'es aux vertices et aux
arr\^etes du maillages, plut\^ot que propres \`a un \'element ou
l'autre.

Cependant, les deux points de vues sont utiles, puisque l'on assemble
la matrice \'element fini \'element par \'element, et non fonction de
base apr\`es fonction de base.

La contribution d'un \'element $K$ au syst\`eme lin\'eaire est d'abord
calcul\'ee en utilisant une num\'erotation des degr\'es de libert\'e
locale \`a $K$, puis la contribution est assembl\'ee dans la matrice
globale. Pour cette \'etape, il est n\'ecessaire d'identifier les
degr\'es de libert\'es locaux aux degr\'es de libert\'es globaux. C'est
le r\^ole de l'espace \'element fini, d\'ecrit plus en d\'etail dans
la section qui suit.

\section{Espace \'elements finis}
On suppose donn\'e un mesh $M$ un \'element fini, c'est-\`a-dire que
$\hat M_v$, $\hat M_a$ et $\hat M_K$ sont donn\'es. Commen\c cons par
enum\'erer l'ensemble des degr\'es de libert\'es $M_\text{tot}$
associ\'es au maillage: 
\begin{align}
  & M_v = \hat M_v N_v\\
  & M_a = \hat M_a N_a\\
  & M_K = \hat M_K N_K\\
  & M_\text{tot} = \hat M_v N_v + \hat M_a N_a + \hat M_K N_K.
\end{align}

On note les degr\'es de libert\'e $\{\varphi_i\}_{i=1}^{M_\text{tot}}$,
et on associe les trois intervalle d'indices avec les degr\'es
associ\'es respectivement aux vertices, aux arr\^etes et aux
\'elements: 

\begin{itemize}
\item Indices des dofs associ\'es aux vertices: $\{1, \dots, \hat M_v
  N_v\}$,
\item Indices des dofs associ\'es aux arr\^etes: $\{\hat M_v N_v + 1,
  \hat M_v N_v + \hat M_a N_a\}$,
\item Indices des dofs associ\'es aux \'elements: $\{\hat M_v N_v + \hat
  M_a N_a + 1, \dots, \hat M_v N_v + \hat M_a N_a + \hat M_K N_K\}$,
\end{itemize}

En particulier, le $\hat i$-\`eme dof associ\'e au vertex $j$ a l'indice global 
\begin{equation}
  i = j \hat M_v + \hat i,
\end{equation}
le $\hat i$-\`eme dof associ\'e \`a l'arr\^ete $j$ a l'indice global 
\begin{equation}
  i = \hat M_v N_v + j \hat M_a + \hat i,
\end{equation}
and the $\hat i$-\`eme dof associ\'e au triangle j a l'indice global 
\begin{equation}
  i = \hat M_v N_v + \hat M_a N_a + j \hat M_K + \hat i.
\end{equation}

\section{Cellule de r\'ef\'erence}
On d\'efinit ici la cellule (triangle) de r\'ef\'erence, et en
particulier les enumeration des differents sous-domaines (vertices,
arr\^etes et triangle). Soit les vertices $v_0 = (0,0)$, $v_1 =
(1,0)$ and $v_2 = (0,1)$, et le triangle est definit par $K = \{v_0,
v_1, v_2\}$. Donc on d\'efinit et enum\`ere les sous domaines de la
mani\`ere suivante: 
\begin{align*}
  \text{vertices: } & v_0, v_1, v_2,\\
  \text{arr\^etes: } & e_0 = \{v_0, v_1\},\ e_1 = \{v_0, v_2\},\ e_2 =
  \{v_1, v_2\}, \\
  \text{triangle: } & K = \{v_0, v_1, v_2\}.
\end{align*}

Il y a quand meme un truc. Il faut que les dof definis sur les
arr\^etes soient enum\'er\'es dans le m\^eme ordre sur chaque arr\^ete de
la cellule de r\'ef\'erence. La question ne se pose pas pour les dofs
d\'efinis sur les \'elements, parce qu'ils ne sont par partag\'es par
diff\'erents \'elements. Pour les dofs d\'efinis sur les vertices, la
question ne se pose pas, parce qu'il y a qu'une seule possibilit\'e
d'enum\'erer un unique objet (le vertex). 

\section{Assemblage}
On consid\`ere le probl\`eme suivant. Soit $V_h$ l'espace \'element finit lagrange $\mathbb P_1\bigcap H^1_0(\Omega)$ sur une triangulation $\tau_h$ de $\Omega$. On cherche la fonction $u_h\in V_h$ telle que:
\begin{align}
  \int_\Omega \nabla u_h\cdot \nabla v = \int_\Omega f v,\quad \forall v \in V_h.
\end{align}
Bien entendu, on prend $v = \phi_j$ les fonctions de base de $V_h$, et on d\'ecompose les int\'egrale sur chaque \'element de la triangulation:
\begin{equation}
  \sum_{K\in\tau_h}\int_K \nabla u_h\cdot \nabla v = \sum_{K \in \tau_h}\int_\Omega f v, \quad \forall v \in V_h.
\end{equation}
L'inconnue $u_h$ s'\'ecrit comme une combinaison lin\'eaire des fonctions de base $\phi_i$:
\begin{equation}
  u_h(x) = \sum_{i = 1}^{M_\text{tot}} u_i \phi_i(x).
\end{equation}
Le syst\`eme devient alors:
\begin{equation}
  \sum_{i = 1}^{M_\text{tot}}u_i\sum_{K\in\tau_h}\int_K \nabla \phi_i\cdot \nabla \phi_j = \sum_{K \in \tau_h}\int_\Omega f \phi_j, \quad \forall j \in \{1, \dots,M_\text{tot}\}.
\end{equation}

Puisque le support des $\phi_j$ sont limit\'es aux \'el\'ements qui partagent le degr\'e de libert\'e $j$, la plupart des int\'egrales sont nulles. Soit $a$ la forme bilin\'eaire d\'efinie par
\begin{equation}
a(u, v) = \sum_{K\in\tau_h}\int_K \nabla u\cdot \nabla v.
\end{equation}
On veut calculer l'ensemble des nombres $a(\phi_i, \phi_j)$:
\begin{equation}
a(\phi_i, \phi_j) = \sum_{K\in\tau_h}\int_K \nabla \phi_i\cdot \nabla \phi_j = \sum_{K\in\tau_h}\int_{\hat K} \det(J(\hat x)) [J^{-\mathrm t}(\hat x)]_{kn} [J^{-\mathrm t}(\hat x)]_{km} \hat\partial_n \hat\phi_i(\hat x)\hat\partial_m \hat\phi_j(\hat x).
\end{equation}
En pratique, les int\'egrales sont remplac\'ees par des quadratures num\'eriques. Donc la forme bilin\'eaire $a$ devient:
\begin{equation}
a(\phi_i, \phi_j) = \sum_{K\in\tau_h} Q \sum_{q = 1}^{n_q}\omega_q \det(J(\hat x_q)) [J^{-\mathrm t}(\hat x_q)]_{kn} [J^{-\mathrm t}(\hat x_q)]_{km} \hat\partial_n \hat\phi_i(\hat x_q)\hat\partial_m \hat\phi_j(\hat x_q),
\end{equation}
avec $Q$ un coefficient de normalisation de la formule de quadrature, $\omega_q$ les poids de la formule de quadrature, et $\hat x_q$ les points de quadrature dans l'\'el\'ement de r\'ef\'erence.

